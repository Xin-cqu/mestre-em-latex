\chapter{Outro assunto interessante}\label{cap3}

\section{Introdução}\label{cap3:intro}

De modo geral, populações de equinóides de águas rasas apresentam ciclo reprodutivo anual com períodos discretos de gametogênese e liberação de gametas \citep{Pearse1991}.
\citet{Lessios1987} observou que o diâmetro dos óvulos de \subdeshort\ do Caribe é menor depois de setembro, mas este padrão não se repetiu no ano seguinte, havendo diferença entre as médias de outubro de 1982 e 1985.

\section{Materiais e Métodos}\label{cap3:mem}

\subsection{Coleta dos adultos}\label{cap3:mem:coleta}

Medimos a altura do eixo \nicefrac{oral}{aboral}, a espessura da borda no ambúlacro III e o peso úmido dos exemplares (precisão de \unit[0,1]{g}).

\subsection{Dissecção}\label{cap3:mem:diss}

Os lotes mensais foram mantidos em baldes de \unit[30]{L} até o processo de dissecção.

Calculamos o índice gonadal através da equação \citep[adaptado de][]{MacCord2004}, baseado no peso úmido de apenas uma gônada de cada indivíduo:
\begin{equation}\label{eq:ind}\notag
IG=\frac{\text{peso úmido da gônada}}{\text{peso úmido do exemplar - (peso úmido da gônada)}}
\end{equation}

Por meio do programa de processamento de imagens \emph{ImageJ} \citep{Rasband1997} calculamos a área, diâmetro, perímetro dos túbulos e área ocupada pelo epitélio germinativo (fagócitos nutritivos e células germinativas).
Os testes foram considerados estatisticamente significativos quando $P<0,050$.
Utilizamos a linguagem \emph{R} \citep{R2005} para o manejo dos dados morfométricos e execução de análises estatísticas.
Em todo o trabalho, os nomes dos testes estão assinalados em fonte \texttt{monoespaçada} e os nomes dos estágios gonadais estão assinalados em \textbf{negrito}.
Os dados estão expressos na forma $\text{média} \pm \text{desvio padrão}$.

\section{Resultados}\label{cap3:res}

\subsection{Anatomia}\label{cap3:res:anato}

A média do peso de uma gônada de \subdeshort\ foi de $2,7\pm\unit[1,1]{g}$ ($n=119$), com amostras entre \unit[0,6]{g} e \unit[7,7]{g}, coletados nos estágios de \textbf{recuperação} e \textbf{prematuro}, apesar das variâncias serem significativamente diferentes ($P=0,007$), os estágios apresentaram distribuição normal.

\subsection{Histologia}\label{cap3:res:histo}

Os túbulos que constituem o tecido gonadal estão imersos no líquido do celoma perivisceral.
A caracterização dos estágios gonadais foi baseada em estudos anteriores realizados com equinodermos \citep[e.g.,][]{Lane1979,Byrne1990,Pearse1991,MacCord2004,Walker2005,Tavares2006}.

\subsubsection{Fêmeas}\label{cap3:res:femeas}

Identificamos 6 estágios de maturação gonadal nas fêmeas de \subdeshort.

\begin{description}
  \item[Recuperação $(n=7)$:] Epitélio germinativo com vilosidades, ocupando $80,4\pm\unit[9,4]{\%}$ da área gonadal.
    Grânulos dos fagócitos nutritivos relativamente densos e presentes na maior parte do citoplasma, com poucos glóbulos hialinos e vacúolos.
    Pequenos oócitos pré-vitelogênicos presentes na lâmina basal; lúmen não contém óvulos.
    O diâmetro médio do túbulo em corte transversal é de $0,68\pm\unit[0,21]{mm}$.
  \item[Proliferação $(n=1)$:] Fagócitos nutritivos contêm grande quantidade de grânulos densos, poucos glóbulos hialinos e vacúolos.
    Muitos oócitos pré-vitelogênicos e vitelogênicos crescendo entre os fagócitos nutritivos; lâmina basal com muitas vilosidades.
    O único túbulo em corte transversal tinha diâmetro reduzido (\unit[0,36]{mm}).
    A ocupação do epitélio germinativo no restante da gônada é semelhante aos estágios de \textbf{recuperação} e \textbf{prematuro}.
    Não há óvulos no lúmen.
\end{description}

\paragraph{Oogênese}

O estágio com maior quantidade de grânulos PAS+ no citoplasma dos fagócitos nutritivos foi o de \textbf{proliferação}.

\begin{description}
  \item[Oogônias:] Citoplasma acidófilo, sem grânulos ou vacúolos e com aparência lisa.
  Diâmetro celular varia entre 4 e \unit[11]{µm}.
  Núcleo apresenta grumos de cromatina e nucléolo irregular.
  A razão $\frac{\text{núcleo}}{\text{citoplasma}}\times100=51,0\pm\unit[11,9]{\%}$.
  Membrana do núcleo bem definida e espessa.
  Célula aderida à lâmina basal comumente agregada à outras oogônias (Figura~\ref{fig:oogenese}a).
  \item[Oócitos primários] ~
    \begin{description}
      \item[Pré-Vitelogênico:] Diâmetro celular de 25 a \unit[75]{µm}.
	A razão $\frac{\text{núcleo}}{\text{citoplasma}}$ torna-se menor ($22,7\pm\unit[5,4]{\%}$), devido ao aumento da área citoplasmática em relação ao núcleo.
	Núcleo de aparência irregular e membrana mais fina.
	Nucléolo torna-se redondo e bastante denso.
	Citoplasma continua liso, sem grânulos ou vacúolos, até certo tamanho.
	Oócitos maiores apresentam pequenas inclusões PAS+ e pequenos vacúolos.
	Membrana sem microvilosidades.
	Lâmina basal coberta por fagócitos nutritivos formando uma câmara de incubação individual; oócitos aderidos à lâmina basal (Figura~\ref{fig:oogenese}b).
      \item[Vitelogênico:] Núcleo com pouca heterocromatina e com membrana fina.
	Nucléolo continua redondo e denso.
	Razão $\frac{\text{núcleo}}{\text{citoplasma}}$ $14,0\pm\unit[1,5]{\%}$; diâmetro das células variando de 94 a \unit[129]{µm}.
	Citoplasma rugoso com inúmeras inclusões densas e vacúolos.
	Grânulos PAS+ em todo o citoplasma.
	Membrana plasmática com microvilosidades.
	Câmara de incubação bastante evidente e oócitos aderidos à lâmina basal (Figura~\ref{fig:oogenese}c,d).
    \end{description}
  \item[Óvulos:] Núcleo raramente visível, com cromatina descondensada e sem nucléolo.
    Quebra da vesícula germinativa (núcleo) parece ocorrer com o oócito ainda aderido à lâmina basal.
    Grânulos PAS+ densamente distribuídos pelo citoplasma exceto na periferia, onde encontram-se os grânulos corticais.
    A migração dos grânulos corticais à periferia da célula parece ocorrer após a quebra da vesícula germinativa.
    Citoplasma bastante vacuolado e rugoso e membrana plasmática sem vilosidades.
    O diâmetro dos óvulos é de 151 $\pm$ \unit[8]{µm}.
\end{description}

\begin{figure}[htbp]%
  \centering%
  \subfloat[Oogônias]{\label{fig:eo01}\includegraphics[width=200pt]{eo01}}\vspace{11pt}
  \subfloat[Oócitos primários pré-vitelogênicos]{\label{fig:eo02}\includegraphics[width=200pt]{eo02}}\\
  \vspace{-18pt}
  \subfloat[Oócito primário vitelogênico inicial]{\label{fig:eo03}\includegraphics[width=200pt]{eo03}}\vspace{11pt}
  \subfloat[Oócito primário vitelogênico tardio]{\label{fig:eo04}\includegraphics[width=200pt]{eo04}}%
  \caption[Oogênese]{Etapas da oogênese em \subdeshort. \subref{fig:eo01}~Oogônias (\textbf{og}) agregadas na lâmina basal. \subref{fig:eo02}~Oócitos primários pré-vitelogênicos (\textbf{oppv}) envolvidos pelos fagócitos nutritivos (\textbf{fn}). \subref{fig:eo03}~Oócito primário vitelogênico inicial (\textbf{opv}) aderido à lâmina basal (\textbf{lb}) de uma vilosidade; citoplasma com algumas inclusões. \subref{fig:eo04}~Oócitos primários vitelogênicos no final da maturação dentro da câmara incubadora (\textbf{ci}) formada pelos fagócitos nutritivos; citoplasma repleto de vitelo. \textbf{sg}, seio genital; \textbf{ln}, lúmen; \textbf{em}, epitélio muscular.}%
  \nomenclature{og}{oogônia}%
  \nomenclature{lb}{lâmina basal}%
  \nomenclature{ci}{câmara incubadora}
  \label{fig:oogenese}%
\end{figure}%

\subsubsection{Morfometria}\label{cap3:res:morf}

Como os conjuntos de valores apresentam distribuição normal e variâncias homogêneas utilizamos o teste \texttt{t} para compará-los (Tabela~\ref{tab:areap}.

\begin{table}[htbp]
  \caption[Valores de $P$ entre médias da área ocupada pelo epitélio germinativo]{Valores de $P$ obtidos na comparação da porcentagem da área do túbulo ocupada pelo epitélio germinativo, através do teste \texttt{t} entre os diferentes estágios gonadais. Valores em negrito foram considerados significativos, $\alpha=0,05$.}
  \label{tab:areap}
  \vspace{1em}
  \centering
  \begin{tabular}{l r r r r r}
    \toprule
     ~			&	Recuperação	&	Proliferação	&	Prematuro	&	Maduro\\
     \midrule
     Proliferação	&	1,000		&	-		&	-		&	-\\
     Prematuro		&	0,883		&	1,000		&	-		&	-\\
     Maduro		&	\textbf{<0,001}	&	\textbf{<0,001}	&	\textbf{<0,001}	&	-\\
     \bottomrule
   \end{tabular}
 \end{table}

No caso, o modelo de regressão linear é descrito pela equação $y=0,799x+0,699$.
Com o intuito de retirar o efeito do tamanho dos túbulos na análise da relação entre área total e área do epitélio germinativo, os resíduos da regressão entre as duas variáveis foram utilizados.

\subsection{Ciclo Reprodutivo}\label{cap3:res:ciclo}

Nos meses de dezembro de 2006, janeiro e fevereiro de 2007 predominaram os estágios \textbf{prematuro} e \textbf{maduro}, os quais apareceram em freqüência menor a partir de fevereiro.
De março a maio predominou o estágio de \textbf{pós-liberação}, quando então as primeiras ocorrências do estágio de \textbf{recuperação} foram registradas.

\section{Discussão}\label{cap3:disc}

A substância negra viscosa encontrada envolvendo as gônadas recém dissecadas de \subde\ é possivelmente composta por agregados de celomócitos com grânulos negros.
A organização dos epitélios e espaços observada nas gônadas de \subdeshort\ foi a mesma generalizada por~\citet[pg.~529]{Pearse1991}, sendo composta de 3 epitélios, uma camada colágena e 3 espaços distintos.
Este padrão não é incomum, havendo diversos equinodermos em que a correlação positiva entre o tamanho corpóreo e peso gonadal, presente durante a fase juvenil, não se mantém nos indivíduos adultos \citep[e.g.,][]{Lane1979,Pearse1991,Tavares2006}.

%\section{Conclusões}\label{ciclo:conc}
%Lista de conclusões
