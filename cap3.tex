\chapter{Outro assunto interessante}\label{cap3}

\section{Introdução}\label{cap3:intro}

Mais uma introdução.

\section{Materiais e Métodos}\label{cap3:mem}

\subsection{Coleta dos adultos}\label{cap3:mem:coleta}

Outro exemplo de como usar frações no eixo \nicefrac{oral}{aboral}.

\subsection{Dissecção}\label{cap3:mem:diss}

Um dos pontos fortes do \LaTeX\ é a praticidade e beleza das fórmulas matemáticas:
\begin{equation}\label{eq:ind}\notag
IG=\frac{\text{peso úmido da gônada}}{\text{peso úmido do exemplar - (peso úmido da gônada)}}
\end{equation}

Usamos o programa de processamento de imagens \emph{ImageJ} \citep{Rasband1997} e a linguagem \emph{R} \citep{R2005} para a morfometria ($P<0,050$).
Os testes estão em fonte \texttt{monoespaçada}, os estágios em \textbf{negrito} e os dados na forma $\text{média} \pm \text{desvio padrão}$.

\section{Resultados}\label{cap3:res}

\subsection{Histologia}\label{cap3:res:histo}

% A inserção de fórmulas no meio do texto requer o uso de $fórmula$
A média foi de $2,7\pm\unit[1,1]{g}$ ($n=119$), com amostras entre \unit[0,6]{g} e \unit[7,7]{g} e $P=0,007$.

\subsubsection{Fêmeas}\label{cap3:res:femeas}

% Outro ambiente útil é o descritpion, como exemplificado abaixo.
\begin{description}
  \item[Estágio1 $(n=27)$:] Descrição minuciosa deste estágio.
    Estou incluindo um pouco de texto extra para mostrar como a formatação fica impecável.
    Uma boa formatação não distrai o leitor e proporciona maior clareza e prazer durante a leitura.
  \item[Estágio2 $(n=25)$:] Descrição minuciosa deste estágio.
    Estou incluindo um pouco de texto extra para mostrar como a formatação fica impecável.
    Uma boa formatação não distrai o leitor e proporciona maior clareza e prazer durante a leitura.
\end{description}

As descrições também podem ser colocadas uma dentro da outra.

\begin{description}
  \item[Tipo1:] Descrição minuciosa deste tipo.
    Estou incluindo um pouco de texto extra para mostrar como a formatação fica impecável.
    A razão $\frac{\text{núcleo}}{\text{citoplasma}}\times100=51,0\pm\unit[11,9]{\%}$.
  \item[Tipo2] ~
    \begin{description}
      \item[Subtipo2.1:] Descrição minuciosa deste tipo.
	Estou incluindo um pouco de texto extra para mostrar como a formatação fica impecável.
      \item[Subtipo2.2:] Descrição minuciosa deste tipo.
	Estou incluindo um pouco de texto extra para mostrar como a formatação fica impecável.Núcleo com pouca heterocromatina e com membrana fina.
    \end{description}
  \item[Tipo3:] Descrição minuciosa deste tipo.
	Estou incluindo um pouco de texto extra para mostrar como a formatação fica impecável.
\end{description}

\subsubsection{Outra tabela}\label{cap3:res:morf}

Outra tabela de exemplo onde utilizamos o teste \texttt{t} (Tabela~\ref{tab:areap}).

\begin{table}[htbp]
  \caption[Tabelas com valores de $P$]{Um exemplo de tabela comum em trabalhos científicos mostrando valores de $P$ em uma comparação estatística, $\alpha=0,05$.}
  \label{tab:areap}
  \vspace{1em}
  \centering
  \begin{tabular}{l r r r r r}
    \toprule
     ~		&	Estágio1	&	Estágio2	&	Estágio3	&	Estágio4\\
     \midrule
     Estágio2	&	1,000		&	-		&	-		&	-\\
     Estágio3	&	0,883		&	1,000		&	-		&	-\\
     Estágio4	&	\textbf{<0,001}	&	\textbf{<0,001}	&	\textbf{<0,001}	&	-\\
     \bottomrule
   \end{tabular}
 \end{table}

No caso, o modelo de regressão linear é descrito pela equação $y=0,799x+0,699$.

\section{Discussão}\label{cap3:disc}

Uma discussão discutindo os resultados.

%\section{Conclusões}\label{ciclo:conc}
%Lista de conclusões
