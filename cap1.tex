\chapter{Introdução}\label{intro}
\section{Histórico da linha de pesquisa}\label{intro:historico}

% O comando \label{} define o nome da parte especificada.
% Você pode citar esta seção usando o comando \ref{}.
% O "~" evita com que ocorra uma quebra de linha entre as palavras.

Neste modelo o Capítulo~\ref{intro} é uma introdução ao contexto do projeto com revisão da literatura.
Aqui vou exemplificar alguns comandos básicos e úteis para uma dissertação, mas gosto mesmo é de evolução \citep{Buss1987,Kutschera2004}.

Um dos pacotes mais interessantes para lidar com citações é o \texttt{natbib}, pois é bastante configurável \citep[ver detalhes sobre as reviravoltas da evolução em][]{Buss1987,Arthur2002}.

\section{Contextualizando melhor}\label{intro:contexto}

\citet{Hart2002} fala bem dos equinodermos e nos mostra como citar espécies modelo como a \emph{Drosophila melagonaster} e \emph{Caenorhabditis elegans}.

\section{Especificidades}\label{intro:especs}

Mencionei na seção~\ref{intro:contexto} como referênciar um capítulo e agora vou fazer de novo explicando que dividi em 2 capítulos para facilitar a apresentação, sendo que o Capítulo~\ref{cap2} trata de um tema (e.g., um artigo) enquanto o Capítulo~\ref{cap3} aborda outro aspecto.
% Este comando faz com que uma página em branco seja incluída após o fim do capítulo. É útil para que todo início de capítulo se inicie em páginas ímpares.
\clearpage{\pagestyle{empty}\cleardoublepage}
