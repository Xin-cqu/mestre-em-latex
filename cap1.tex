\chapter{Introdução}\label{intro}
\section{Histórico da linha de pesquisa}\label{intro:historico}

% O comando \label{nome} define o marcador da parte especificada.
% Você pode citar esta seção usando o comando \ref{nome}.
% O "~" evita uma quebra de linha entre as palavras.

Neste modelo, o Capítulo~\ref{intro} é uma introdução ao contexto do projeto com revisão da literatura.
Aqui vou exemplificar alguns comandos básicos e úteis para uma dissertação, mas gosto mesmo é de evolução \citep{Buss1987,Kutschera2004}.

Um dos pacotes mais interessantes para lidar com citações é o \texttt{natbib}, pois é bastante configurável \citep[ver detalhes em][]{Arthur2002}.

\section{Contextualizando melhor}\label{intro:contexto}

\citet{Hart2002} fala de equinodermos e cita espécies modelo como a \emph{Drosophila~melagonaster} e \emph{Caenorhabditis~elegans}.

Mencionei na seção~\ref{intro:historico} como referênciar um capítulo.
Agora devo explicar que dividi o texto em 2 capítulos, sendo que o Capítulo~\ref{cap2} trata de um tema (e.g., um artigo) enquanto o Capítulo~\ref{cap3} aborda outro aspecto do assunto.

% Este comando faz com que uma página em branco seja incluída após o fim do capítulo.
% Use-o caso você precise criar uma página em branco para que os capítulos se iniciem sempre em páginas ímpares.
%\clearpage{\pagestyle{empty}\cleardoublepage}
