% Capa
\begin{titlepage}
% Os comandos abaixo acertam as distâncias para margens alternadas.
% Este tipo de margem é usado para impressões frente e verso.
\oddsidemargin 1.2cm
\evensidemargin 0cm
% Se quiser uma figura de fundo na capa use o pacote wallpaper e descomente a linha abaixo.
% \ThisCenterWallPaper{0.8}{nomedafigura}
\begin{center}
{\LARGE \nomedoaluno}
\par
\vspace{200pt}
{\Huge \tit}
\par
\vspace{280pt}
\textbf{{\large São Paulo}\\
{\large 2009}}
\end{center}
\end{titlepage}
% Cria uma página em branco, útil na hora de imprimir frente e verso.
\clearpage{\pagestyle{empty}\cleardoublepage}

% Comandos para margens novamente
\oddsidemargin 1.2cm
\evensidemargin 0cm

% Números das páginas em algarismos romanos
\pagenumbering{roman}

\thispagestyle{empty}
% Página de Rosto
\begin{center}
{\LARGE \nomedoaluno}
\par
\vspace{203pt}
{\Huge \tit}
\end{center}

\par
\vspace{90pt}
\hspace*{16em}\parbox{7.6cm}{{\large Dissertação apresentada ao Instituto de Biociências da Universidade de São Paulo, para a obtenção de Título de Mestre em Ciências, na Área de XXXXXXXX.}}

\par
\vspace{1em}
\hspace*{16em}\parbox{7.6cm}{{\large Orientador: Nome do Orientador}}

\par
\vspace{100pt}
\begin{center}
\textbf{{\large São Paulo}\\
{\large 2009}}
\end{center}

\newpage

% Ficha Catalográfica
\thispagestyle{empty}

\hspace{8em}\fbox{\begin{minipage}{10cm}
Aluno, Nome C.

\hspace{2em}\tit

\hspace{2em}\pageref{LastPage} páginas

\hspace{2em}Dissertação (Mestrado) - Instituto de Biociências da Universidade de São Paulo. Departamento de XXXXXXXX.

\begin{enumerate}
\item Palavra-chave
\item Palavra-chave
\item Palavra-chave
\end{enumerate}
I. Universidade de São Paulo. Instituto de Biociências. Departamento de XXXXXXXX.

\end{minipage}}
\par
\vspace{2em}
\begin{center}
{\LARGE\textbf{Comissão Julgadora:}}

\par
\vspace{10em}
\begin{tabular*}{\textwidth}{@{\extracolsep{\fill}}l l}
\rule{16em}{1px} 	& \rule{16em}{1px} \\
Prof. Dr. 		& Prof. Dr. \\
Nome			& Nome
\end{tabular*}

\par
\vspace{10em}

\parbox{16em}{\rule{16em}{1px} \\
Prof. Dr. \\
Nome do Orientador}
\end{center}

\newpage

% Dedicatória
\thispagestyle{empty}

% Posição do texto na página
\vspace*{0.75\textheight}
\begin{flushright}
  \emph{Dedicatória...}
\end{flushright}

\newpage

% Epígrafe
\thispagestyle{empty}

\vspace*{0.4\textheight}
\noindent{\LARGE\textbf{Exemplo de epígrafe}}
% Tudo que você escreve no verbatim é renderizado literalmente (comandos não são interpretados e os espaços são respeitados)
\begin{verbatim}
O que é bonito?
É o que persegue o infinito;
Mas eu não sou
Eu não sou, não…
Eu gosto é do inacabado,
O imperfeito, o estragado, o que dançou
O que dançou…
Eu quero mais erosão
Menos granito.
Namorar o zero e o não,
Escrever tudo o que desprezo
E desprezar tudo o que acredito.
Eu não quero a gravação, não,
Eu quero o grito.
Que a gente vai, a gente vai
E fica a obra,
Mas eu persigo o que falta
Não o que sobra.
Eu quero tudo que dá e passa.
Quero tudo que se despe,
Se despede, e despedaça.
O que é bonito…
\end{verbatim}
\begin{flushright}
Lenine e Bráulio Tavares
\end{flushright}

\newpage

% Agradecimentos
\thispagestyle{empty}

% Espaçamento duplo
\doublespacing

\noindent{\LARGE\textbf{Agradecimentos}}

Agradeço ao meu orientador, ao meu co-orientador, aos meus colaboradores, aos técnicos, à seção administrativa, à fundação que liberou verba para minhas pesquisas, aos meus amigos, à minha família e ao meu grande amor.

\thispagestyle{empty}
\newpage

\begin{abstract}
\label{resumo}
% Pode parecer estranho, mas colocar uma frase por linha ajuda a organizar e reescrever o texto quando necessário.
% Além disso, ajuda se você estiver comparando versões diferentes do mesmo texto.
% Para separar parágrafos utilize uma linha em branco.
Esta, quem sabe, é a parte mais importante do seu trabalho.
É o que a maioria das pessoas vai ler (além do título).
Seja objetivo sem perder conteúdo.

Um bom resumo explica porquê este trabalho é interessante, relata como foi feito, o que foi encontrado, contextualiza os resultados e delineia conclusões.
\end{abstract}

% Criei a página do abstract na mão, por isso tem bem mais comandos do que o resumo acima, apesar de serem idênticas.
\vspace*{10pt}
% Abstract
\begin{center}
  \emph{\begin{large}Abstract\end{large}}\label{abstract}
\vspace{2pt}
\end{center}

% Selecionar a linguagem acerta os padrões de hifenação diferentes entre inglês e português.
\selectlanguage{english}
\noindent
This is the most important part of your work.
This is what most people will read.
Be concise without omitting content.

A good abstract explains why this is an interesting study, tells how it was done, what was found, contextualizes the results and set conclusions.

% Voltando ao português...
\selectlanguage{brazilian}

\newpage

% Números das páginas em arábicos
\pagenumbering{arabic}

% Índice
\tableofcontents

% Lista de figuras
\listoffigures

% Lista de tabelas
\listoftables

% Abreviações
% Para imprimir as abreviações siga as instruções em 
% http://code.google.com/p/mestre-em-latex/wiki/ListaDeAbreviaturas
\printnomenclature
