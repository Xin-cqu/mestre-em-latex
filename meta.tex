%% Pacotes utilizados
% Codificação e formatação básica do LaTeX
\usepackage[english,brazilian]{babel}		% Suporte para português (hifenação e caracteres especiais)
\usepackage[utf8]{inputenc}			% Codificação do arquivo
\usepackage{cmap}				% Mapear caracteres especiais no PDF
\usepackage[T1]{fontenc}			% Codificação da fonte
\usepackage{amsmath,amssymb,amsfonts,textcomp}	% Essencial para colocar funções e outros símbolos matemáticos

% Layout
\usepackage[twoside]{geometry}			% Customização do layout da página, margens espelhadas
\usepackage{setspace}				% Para definir espaçamento entre as linhas
\setlength{\fboxsep}{1em}			% Espaçamento do texto para o frame
%\geometry{hcentering}				% Faz com que as margens tenham o mesmo tamanho horizontalmente

% Elementos Gráficos
\usepackage[]{graphicx}				% Para incluir figuras (pacote extendido)
\usepackage{color}				% Suporte a cores
\usepackage{subfig}				% Criar figura dividida em subfiguras
\graphicspath{{./figuras/}}			% Caminho para o diretório das figuras
\usepackage{caption}				% Customizar as legendas de figuras e tabelas
\captionsetup[subfigure]{style=default, margin=0pt, parskip=0pt, hangindent=0pt, indention=0pt, singlelinecheck=true, labelformat=parens, labelsep=space}
\usepackage{multicol}				% Criar ambientes com 2 ou mais colunas
%\usepackage{wallpaper}				% Colocar figuras de fundo (e.g., capa)
%\addtolength{\wpYoffset}{-140pt}		% Ajuste da posição da figura no eixo Y
%\addtolength{\wpXoffset}{36pt}			% Ajuste da posição da figura no eixo X

% Tabelas
\usepackage{array}				% Elementos extras para formatação de tabelas
\usepackage{booktabs}				% Tabelas com qualidade de publicação
\usepackage{longtable}				% Para criar tabelas maiores que uma página
\usepackage{lscape}				% adicionar tabelas e figuras como landscape

% Lista de Abreviações
\usepackage[intoc,portuguese]{nomencl}		% Cria lista de abreviações
\makenomenclature				% Comando para criar

% Notas de rodapé
\usepackage{footnote}				% Lidar com notas de rodapé em diversas situações
\makesavenoteenv{tabular}			% Notas criadas nas tabelas ficam no fim das tabelas

% Links dinâmicos
\usepackage{hyperref}				% Suporte para hipertexto, links para referências e figuras
\hypersetup{colorlinks=true, linkcolor=blue, citecolor=blue, filecolor=blue, pagecolor=blue, urlcolor=green,
            pdfauthor={Nome do Autor},
            pdftitle={Título do Projeto},
            pdfsubject={Assunto do Projeto},
            pdfkeywords={palavra-chave, palavra-chave, palavra-chave},
            pdfproducer={Latex},
            pdfcreator={pdflatex}}		% Configurações dos links e metatags do PDF a ser gerado
\usepackage{lastpage}				% Conta o número de páginas

% Referências bibliográficas e afins
\usepackage{natbib}				% Formatar as citações no texto e a lista de referências
\usepackage[nottoc]{tocbibind} 			% Adicionar bibliografia, índice e conteúdo na Tabela de conteúdo

% Pontuação e unidades
\usepackage{icomma}				% Posicionar inteligentemente a vírgula como separador decimal
\usepackage[tight]{units}			% Formatar as unidades com as distâncias corretas

% Cabeçalho e rodapé
\usepackage{fancyhdr}				% Controlar os cabeçalhos e rodapés
\pagestyle{fancy}				% Usar os estilos do pacote fancyhdr
\fancypagestyle{plain}{\fancyhf{}}
\fancyhead{}					% Limpar os campos do cabeçalho atual
\fancyhead[LO,RE]{\thepage}			% Número da página do lado esquerdo [L] nas páginas ímpares [O] e do lado direito [R] nas páginas pares [E]
\fancyhead[RO]{\nouppercase{\rightmark}}	% Nome da seção do lado direito em páginas ímpares
\fancyhead[LE]{\nouppercase{\leftmark}}		% Nome do capítulo do lado esquerdo em páginas pares
\fancyfoot{} 					% Limpar os campos do rodapé
\renewcommand{\headrulewidth}{0pt}		% Omitir linha de separação entre cabeçalho e conteúdo
\renewcommand{\footrulewidth}{0pt}		% Omitir linha de separação entre rodapé e conteúdo
\headheight 13.6pt				% Altura do cabeçalho

% Inserir comentários no texto
\usepackage[margins]{trackchanges}		% Marcar mudanças e fazer comentários
\renewcommand{\initialsTwo}{bcv}		% Iniciais do autor
\reversemarginpar				% Notas na margem interna


%% Comandos customizados

% Espécie estudada
\newcommand{\subde}{\emph{Clypeaster~subdepressus}}	% Espécie
\newcommand{\subdeshort}{\emph{C.~subdepressus}}	% Espécie abreviada

% Título do projeto
\newcommand{\tit}{Título original do projeto que normalmente é o que você definiu lá no começo e não pode mais mudar}
\newcommand{\nomedoaluno}{Nome Completo do Aluno}

%\newcommand{\gepeto}[1]{\vspace*{12pt}\begin{singlespace}\hspace{2cm}\begin{small}\parbox{12cm}{#1}\end{small}\end{singlespace}\vspace*{20pt}}

% Novo ambiente
%\newenvironment{gepeto}[1]
%{\vspace{1em}\begin{singlespace}\noindent\ignorespaces\hspace*{1cm}\parbox{3cm}{#1}}
%{\vspace{1em}\end{singlespace}}
%% Pacotes não implementados

%\usepackage[fixlanguage]{babelbib}		% Escrever referências em outras línguas que não inglês
%\usepackage{abntcite}				% Formatação de citações e referências segundo a ABNT
%\usepackage{xindy}				% Índice remissivo

%\usepackage{biocon}				% Para lidar com nomes de espécies [não implementado]
%\newanimal{Cs}{genus=Clypeaster,epithet=subdepressus,author=Gray,year=1825}
%\newcommand{\subde}{\animal[e]{Cs}}		% Comando para escrever a espécie por extenso [não implementado]
%\newcommand{\subdeshort}{\animal[a]{Cs}}	% Comando para escrever a espécie abreviada [não implementado]
