\chapter{Um assunto legal}\label{cap2}

\section{Introdução}\label{cap2:intro}

O modelo novo sugere que cada capítulo tenha um resumo, introdução, materiais e métodos, resultados, discussão e conclusões, mas optei por deixar o resumo de lado.

\section{Materiais e Métodos}\label{cap2:mem}

\subsection{Coleta dos organismos}\label{cap2:mem:coleta}

Você pode dividir cada seção em subseções para organizar melhor o conteúdo.

\subsection{Cultivo dos seres vivos}\label{cap2:mem:gametas}

\subsubsection{Embriões e Larvas}

Você também pode criar subsubseções caso necessário.
% Para inserir fórmulas matemáticas bonitinhas coloque $entre$.
% Note como formatar a unidade de temperatura e como utilizar o comando \nicefrac para frações.
A cultura foi mantida num ciclo de $12:12$ horas e temperatura constante de \unit[24]{°C}; a concentração final foi de $8\times10^5$ e $1\times10^6$ \nicefrac{células}{mL}.

\subsubsection{Metamorfose}

% Como o % representa um comentário e não é compilado, para fazê-lo aparecer no texto você precisa colocar uma "\" antes, como abaixo.
Apenas \unit[4]{\%} do texto está contido em subsubseções.

\subsection{Microscopia de luz}\label{cap2:mem:micro}

As amostras também foram colocadas em placas $35\times\unit[10]{mm}$ para visualização na lupa ou através de câmera de vídeo conectada via porta \emph{Firewire}.

\section{Resultados}\label{cap2:res}

\subsection{Primeiras figuras}\label{cap2:res:figs}

Subseções de novo, mas é hora de colocar algumas figuras para mostrar resultados (Figura~\ref{fig:pronuc}).

% Veja a documentação para mais detalhes de como inserir figuras
\begin{figure}[htbp]
  \centering
  \includegraphics[width=\textwidth]{pronuc}
  \caption[Figura simples]{Figura abstrata simples com largura igual à largura do texto.}
  \label{fig:pronuc}
\end{figure}

\begin{figure}[htbp]
  \centering
  \includegraphics[width=\textwidth/2]{pronuc}
  \caption[Outra figura simples]{Figura abstrata simples com largura igual à metade da largura do texto.}
  \label{fig:pronuc2}
\end{figure}

Você também pode inserir múltiplas figuras em uma só, permitindo alinhá-las de forma flexível e consistente (Figura~\ref{fig:tufo}).

\begin{figure}[htbp]%
  \centering%
  \subfloat[Oogônias]{\label{fig:t1}\includegraphics[width=200pt]{tufo}}\vspace{11pt}
  \subfloat[Oócitos primários pré-vitelogênicos]{\label{fig:t2}\includegraphics[width=200pt]{tufo}}\\
  \vspace{-18pt}
  \subfloat[Oócito primário vitelogênico inicial]{\label{fig:t3}\includegraphics[width=200pt]{tufo}}\vspace{11pt}
  \subfloat[Oócito primário vitelogênico tardio]{\label{fig:t4}\includegraphics[width=200pt]{tufo}}%
  \caption[Figura com subfloats]{Exemplo de figura com subfloats. \subref{fig:t1}~Oogônias (\textbf{og}) na lâmina. \subref{fig:t2}~Oócitos primários pré-vitelogênicos (\textbf{oppv}) envolvidos. \subref{fig:t3}~Oócito primário vitelogênico inicial (\textbf{opv}) aderido. \subref{fig:t4}~Oócitos primários vitelogênicos dentro da câmara. \textbf{sg}, seio genital; \textbf{ln}, lúmen.}%
  \nomenclature{og}{oogônia}%
  \nomenclature{oppv}{oócitos primários pré-vitelogênicos}%
  \nomenclature{opv}{oócitos primários vitelogênicos}%
  \nomenclature{sg}{seio genital}%
  \nomenclature{ln}{lúmen}
  \label{fig:tufo}%
\end{figure}%

\subsection{Tabelas}\label{cap2:res:tabs}

O desenvolvimento está resumido na Tabela~\ref{tab:exemplo}.

% Para criar tabelas
\begin{table}[htbp]
  \caption[Tabela com \texttt{booktabs}]{Exemplo de legenda de tabela criada com o pacote \texttt{booktabs}.}
  \label{tab:exemplo}
  \vspace{1em}
  \centering
  \begin{tabular}{l l}
    \toprule
    Eventos		&	Tempo\\
    \midrule
    Entrada		&	0\\
    Elevação		&	\unit[40]{s}\\
    Corrida		&	\unit[6]{min}\\
    Saída		&	\unit[15]{min}\\
    \bottomrule
  \end{tabular}
\end{table}

% Usando o pacote units
Ocorrem no eixo \nicefrac{animal}{vegetal} (\nicefrac{A}{V}) e dividem à \unitfrac[500]{µm}{s}.
Desenvolvimento ocorre por volta de \unit[7,5]{h} após a fertilização.
% Usando o pacote nomencl
Após \unit[10]{h}, as células mesenquimais primárias (CMP) iniciam sua ingressão.%
\nomenclature{CMP}{células mesenquimais primárias}
Flagramos as células mesenquimais secundárias (CMS) na blastocele.%
\nomenclature{CMS}{células mesenquimais secundárias}
% Colocando aspas e exemplo do pacote natbib
Larvas apresentam comportamento de ``teste de substrato'' com braços larvais \citep[][pg.~430]{Hyman1955}.

\section{Discussão}\label{cap2:disc}

\citet{Chia1977} verificou papilas em \subdeshort.
A migração (\unitfrac[0,1]{µm}{s}) foi como \emph{C.~japonicus} (\unitfrac[0,082]{µm}{s}).
\emph{Mellita~quinquiesperforata} desenvolve-se \citep{Caldwell1972}\footnote{A temperatura não foi precisada; embriões fecundados entre 25 e \unit[28]{°C}.}.
A evolução deste caráter pode ser vista de duas formas.

\begin{enumerate}
  \item{Múltiplos esferídios $\longrightarrow$ Um esferídio}\label{hipo:1}
    \begin{itemize}
      \item{Um esferídio seria plesiomórfico para Clypeasteroida}
      \item{Dois esferídios seria uma autapomorfia de Clypeasterina}
    \end{itemize}
  \item{Múltiplos esferídios $\longrightarrow$ Dois esferídios $\longrightarrow$ Um esferídio}\label{hipo:2}
    \begin{itemize}
      \item{Dois esferídios seria plesiomórfico para Clypeasteroida}
    \end{itemize}
\end{enumerate}

Segundo \citet{Mooi1990}, não existem critérios para um esferídio [\ref{hipo:2}], e uma alternativa seria o único [\ref{hipo:1}], tendo o fóssil \emph{Togocyamus}.
Jovens cresceram de \unit[0,35]{mm} para \unit[3,80]{mm} no primeiro mês.

%\clearpage{\pagestyle{empty}\cleardoublepage}
